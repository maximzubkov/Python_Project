
\documentclass[a4paper,12pt]{article}
\usepackage[T2A]{fontenc}
\usepackage[utf8]{inputenc}
\usepackage[english,russian]{babel}
\usepackage{amsmath,amsfonts,amssymb,amsthm,mathtools}
\usepackage{wasysym}
\usepackage[utf8]{inputenc}
\usepackage{ tipa }
\usepackage{tikz}
\usetikzlibrary{automata,positioning}
 \usetikzlibrary{calc}
 \usetikzlibrary{patterns}
 \usepackage[linguistics]{forest}
 \usetikzlibrary{decorations.pathreplacing}
 \usepackage{algorithm}
\usepackage{algpseudocode}
\usepackage{xcolor}
\usepackage{hyperref}
\usepackage{amsmath}
\DeclareMathOperator*{\argmax}{arg\,max}
\DeclareMathOperator*{\argmin}{arg\,min}

% Цвета для гиперссылок
\definecolor{linkcolor}{HTML}{799B03} % цвет ссылок
\definecolor{urlcolor}{HTML}{799B03} % цвет гиперссылок
 
\hypersetup{pdfstartview=FitH,  linkcolor=linkcolor,urlcolor=urlcolor, colorlinks=true}
 
 

\title{АМВ}
\author{Зубков Максим, 777 группа}


\begin{document}


\maketitle

\newcommand{\lineann}[5][0.5]{%
    \begin{scope}[rotate=#2, #5,inner sep=2pt]
        \draw[dashed, #5!40] (0,0) -- +(0,#1)
            node [coordinate, near end] (a) {};                       
        \draw[dashed, #5!20] (#3,0) -- +(0,#1)
            node [coordinate, near end] (b) {};
        \draw[|<->|] (a) -- node[fill=white] {#4} (b);
    \end{scope}
        
}
\maketitle


\section{Алгоритм Баума-Вэлша}

Алгоритм Баума-Велша занимается обучаением скрытой марковской модели (далее СММ). Алгоритм итеративно изменяет модель $\lambda = (A, B, \pi)$  таким образом, чтобы вероятность $P(obs | \lambda)$ была максимальной.

Пусть $\lambda$ - текущая модель, а $\widehat{\lambda}$ - кандидат стать новой моделью, необходимо найти такое $\widehat{\lambda}$, чтобы $P(obs |\widehat{\lambda} ) \geq P(obs, \lambda)$ или, что экивалентно $\log{P(obs |\widehat{\lambda} }) \geq \log {P(obs, \lambda)}$. Введем вспомогательную функцию  $Q(\widehat{\lambda} | \lambda) = \mathbb{E}\left[\log P(obs, S | \widehat{\lambda}) | obs, \lambda\right]$ по определению условного мат ожидания  $\sum_{s} P(S | obs, \lambda) \cdot \log \left[P(obs, S | \widehat{\lambda})\right]$. Можно так же доказать, что задча поиска $\widehat{\lambda} = \argmax\limits_{\lambda} \sum_{s} P(obs, S | \lambda)$ эквивалентна задаче поиска $\argmax\limits_{\lambda} Q(\widehat{\lambda} | \lambda)$


\end{document}


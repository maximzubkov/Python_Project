
\documentclass[a4paper,12pt]{article}
\usepackage[T2A]{fontenc}
\usepackage[utf8]{inputenc}
\usepackage[english,russian]{babel}
\usepackage{amsmath,amsfonts,amssymb,amsthm,mathtools}
\usepackage{wasysym}
\usepackage[utf8]{inputenc}
\usepackage{ tipa }
\usepackage{tikz}
\usetikzlibrary{automata,positioning}
 \usetikzlibrary{calc}
 \usetikzlibrary{patterns}
 \usepackage[linguistics]{forest}
 \usetikzlibrary{decorations.pathreplacing}
 \usepackage{algorithm}
\usepackage{algpseudocode}
\usepackage{xcolor}
\usepackage{hyperref}

% Цвета для гиперссылок
\definecolor{linkcolor}{HTML}{799B03} % цвет ссылок
\definecolor{urlcolor}{HTML}{799B03} % цвет гиперссылок
 
\hypersetup{pdfstartview=FitH,  linkcolor=linkcolor,urlcolor=urlcolor, colorlinks=true}
 
 

\title{Примение марковских цепей для распознавания поведения пользователя}
\author{Зубков Максим, Матвей Турков, 777 группа}


\begin{document}


\maketitle

\newcommand{\lineann}[5][0.5]{%
    \begin{scope}[rotate=#2, #5,inner sep=2pt]
        \draw[dashed, #5!40] (0,0) -- +(0,#1)
            node [coordinate, near end] (a) {};                       
        \draw[dashed, #5!20] (#3,0) -- +(0,#1)
            node [coordinate, near end] (b) {};
        \draw[|<->|] (a) -- node[fill=white] {#4} (b);
    \end{scope}
        
}
\maketitle


\section{Введение}

Задача заключается в том, чтобы, имея данные о поведении пользователя в сети, определить, является ли данный пользователь владельцем компьютера или же за компьютером сидит самозванец. С помощью $JS$-плагина мы получаем некоторый набор данных $\vec{x}$, содержащий в себе такие данные как усредние за некоторый промежуток времени  скорость и ускорения мыши, скорость печати, адреса $web$-страниц и так далее. Планируется использовать модель основанную на цепях Маркова.


\section{Цепи Маркова}

\begin{enumerate}

\item Обычные цепи Маркова. Цепь Маркова - это взвешеный граф, переходы между вершинами $u$ и $v$ которого существуют тогда и только тогда, когда пользователь совершал переход между данными вершинами. Вес ребра $(u, v)$ называется веротностью перехода из вершины $u$ в вершину $v$, при этом в момент времени $t$ вероятность перейти из состояния $Q_i$ в состояние $Q_j$ зависит только от веса ребра $(u,v)$ $P(q_{t} = v | q_{t-1} = v, q_{t-2} = ...) = P(q_{t} = v | q_{t-1} = u)$. 

Граф переходов можно задать матрицей $A$, которая, очевидно обладает следующим свойством: пусть мы имеем некоторое начальное начальное распределение веротностей по вершинам графа $\vec{\pi_0}$, тогда мы можем получить распределе на следущем шаге, $\vec{\pi_1}$, умножив матрицу $A$ справа на $\vec{\pi_0}$, тогда по индукции можно докaзать, что на $n$-ом шаге распределение $\vec{\pi_n} = A^{n}\vec{\pi_0}$. С точки зрения алгоритма будет удобнее разложить матрицу $A$ по базису собственных векторов, чтобы уменьшить трудоемкость возведения в степень. 

\item Скрытые цепи Маркова. 

\end{enumerate}
\end{document}

